%\begin{description}
%\item[<ord>]
% <beskrivelse>
%\end{description}

\begin{description}
\item[Legemiddelgjennomgang]
 er en systematisk vurdering av pasientens legemiddelbruk. \citep{Legemiddelverket_LMG}. 
\end{description}

\begin{description}
\item[Legemiddelsamstemming]
er å lage en liste over alle legemidler pasienten bruker. Legemiddelinformasjonen blir hentet fra journaler, intervjuer eller samtaler med fastlege.
\end{description}

\begin{description}
\item[Farmakogenetikk]
er læren om hvordan gener påvirker kroppens opptak av legemidler.
\end{description}


\begin{description}
\item[Legemidler i bruk]
er en liste over legemidler en pasient bruker. Denne listen blir laget under en legemiddelsamstemming.
\end{description}


\begin{description}
\item[Bivirkning]
er en sideeffekt man kan få av et legemiddel eller en annen behandling.
\end{description}

\begin{description}
\item [Interaksjon]
er når to eller flere medisiner påvirker hverandre.
\end{description}
