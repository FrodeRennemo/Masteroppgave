Takk for at du er villig til å delta i forsøket vårt.

Vi er tre IT-studenter ved Institutt for Datateknikk og Informasjonsvitenskap ved NTNU som arbeider med vår avsluttende masteroppgave om datastøttet legemiddelgjennomgang.

I forsøket skal du bruke et elektronisk LMG-skjema som likner på det du er vant til. Du blir presentert for to pasientkasus, og for ett av kasusene vil systemet presentere mulig relevante tiltak eller råd. Testen kunne vært gjennomført på papir, men det gjøres på nett av praktiske hensyn. Etter hvert kasus får du noen spørsmål hvor vi undersøker effekten av datastøtten. Alle deltakere får oppgaver og kasus som likner mye på hverandre, så vær snill å ikke samarbeide eller røpe innholdet før svarfristen er gått ut mandag 9. mai kl. 12:00

Undersøkelsen er anonym. Vi samler ingen personlig informasjon og bruker individuavhengige identifikatorer. Vi trenger dermed ingen godkjenning fra NSD (eller REK).

Svarene fra spørreundersøkelsen vil vurderes av en farmasøyt.


Veiledning i gjennomføring av forsøket:

Forsøket går ut på å lese en kasuistikk, se en samstemt legemiddeloversikt, fylle ut skjemaet med identifiserte LRP og til slutt svare på noen spørsmål.
Du skal gjøre dette to ganger, for to forskjellige pasientkasus.

Innholdet er fordelt på flere nettsider, og du kan bestandig navigere fram og tilbake mellom disse med knappene øverst til venstre («forrige» og «neste»).
(I et virkelig system ville all informasjonen være på ett skjermbilde, men det var upraktisk i et nettbasert eksperiment.)

Når du er ferdig med et pasientkasuset blir du ledet videre til en spørreundersøkelse.
Etter du har gjort LMG og svart på spørreundersøkelse med første brukernavn går du videre til brukkernavn nr 2.
Hele forsøket er forventet å ta ca. 30 minutter.
Tilgang til forsøk:
Brukernavn 1. : storekorn
Brukernavn 2. : finekorn

Trykk på denne lenken for å starte forsøket: http://semlmg.azurewebsites.net

Har du spørsmål?

Kontakt:
Navn: Espen Rise Halstensen, Frode Rennemo og Håvard Moås.
Mail: helsemaster@list.stud.ntnu.no
Tlf: 97 97 45 51


Hovedveileder:
Navn: Øystein Nytrø
Mail: oeystein.nytroe@gmail.com

Biveiledere:
Navn: Ingvild Klevan
Mail: Ingvild.Klevan@sykehusapoteket.no

Navn: Janne Sund
Mail: Janne.Sund@sykehusapoteket.no

