

Legemiddelgjennomgang er en del av legemiddelhåndteringsprosessen ment å kvalitetssikre pasienters legemiddelbruk, ivareta effekt og sikkerhet samt forebygge skader. Ved legemiddelgjennomgang har helsepersonell behov for å finne relevant klinisk informasjon. Mangel på søkeferdigheter og økende volum av informasjon fører til at søken etter informasjon er en tidkrevende og problematisk prosess. Manglende funn av relevant informasjon kan føre til at legemiddelgjennomgangen tar lengre tid og være av lavere kvalitet enn den kunne ha vært. Semantisk webteknologi kan være til støtte ved å gjenbruke data på tvers av applikasjoner, samt aggregere og kombinere data. 

Målet med masteroppgaven var å bruke semantisk webteknologi for å støtte opp om legemiddelgjennomgangsprosessen. Forskningsspørsmålet som ble besvart var følgende:
 \textit{Kan et system  basert på semantisk web-teknolgi hjelpe farmasøyter til å gjøre legemiddelgjennomgang slik at den blir mer presis, raskere gjennomført og lærerik.}

En prototype der legemiddelgjennomgang utføres elektronisk ble utviklet. Prototypen ga brukeren forslag til tiltak der nedsatt nyrefunksjon påvirket pasienters opptak av legemidler. Et eksperiment med bruk av prototypen ble utført av for å besvare forskningsspørsmålet. Eksperimentet sammenlignet utførelsen av legemiddelgjennomgang med og uten forslag til tiltak. Legemiddelgjennomgangen ble utført av kliniske farmasøyter.  

Konklusjonen av eksperimentet var at forslag til tiltak ga mer presis legemiddelgjennomgang og læringsutbytte enn uten forslag. Tidsbruken ble lengre ved bruk av støtten.   

%.\ot{Er dette for detaljert/unødvendig?} En nettside fungerte som brukergrensesnitt for tilhørende kunnskapsbase utviklet med bruk av semantiske webteknologier.
%Legemiddelbruk er et viktig tiltak i helsevesenet, men er uten samlet kontroll og ofte mindre koordinert mellom ulike institusjoner. Suboptimal koordinering mellom helseinstitusjoner kan føre til utilsiktede hendelser som konsekvens av legemiddelrelaterte problemer. En slik systematisk gjennomgang av en pasients legemidler kan eliminere noen av problemene knyttet til et fragmentert helsevesen, og forbedre legemiddelbehandlingen. 