
Det var behov for å utvide kunnskapen vår omkring legemiddelgjennomgang, klinisk farmasi og helsevesenet generelt. Økt kunnskap og eksponering av vår problemstilling ville føre til en bedre forståelse av problemdomenet og gi verdifulle tilbakemeldinger. Denne seksjonen tar for seg vår kontakt med helsevesenet. 

\section{Hospitering ved legemiddelgjennomgang}\label{hospitering}
Våre biveiledere, Janne og Ingvild, introduserte oss for farmasøytenes ansvarsoppgaver på klinikken gjennom flere møter i løpet av høsten 2015. De hjalp oss også med å arrangere et møte med en klinisk farmasøyt på St. Olavs Hospital hvor vi fikk grundig gjennomgang av hvordan farmasøytene jobber på klinikken. Farmasøyten tok oss gjennom et reellt kasus fra klinikken og vi fikk høre hvordan det hele hang sammen. Farmasøyten tok oss gjennom samstemmingen og problematikken rundt pasienter som ikke selv vet hva de skulle tatt av medikamenter og ikke alltid vet hva de tar. Selve legemiddelgjennomgangen og prosessene rundt hvilke kilder hun bruker for å utføre jobben. Og til slutt hvordan resultatet av legemiddelgjennomgangen blir overført til lege og journal.

På våren 2016 ble Espen med en klinisk farmasøyt på hospitering. Hospiteringen foregikk på gastrosenteret ved St. Olavs. Espen ble kastet rett inn i en situasjon der den kliniske farmasøyten hadde fått en henvendelse av en lege for å utføre en legemiddelgjennomgang på en pasient. Farmasøyten hadde kommet frem til et par ting han mente burde endres og leverte denne meldingen muntlig til den aktuelle legen. Meldingen ble overgitt til lege med begrunnelse for anbefalingen. Anbefalingen var endring av dose på et legemiddel og seponering av et annet. Dette var en kort sekvens i starten på hospiteringen, den varte kanskje maks 5minutter. Etter dette satt de seg på et kontor der de snakket om farmasøyten sin jobb. Espen viste også frem en tidlig utgave av prototypen og snakket med farmasøyten om hvilke typer tiltak som burde bli vist til bruker. Farmasøyten brukte \url{www.uptodate.com} som en av kildene da han gjorde legemiddelgjennomgang og anbefalte oss å ta en titt på denne kilden. Mastergruppen tok i bruk denne kilden etter anbefaling fordi den var oversiktlig når det gjaldt nedsatt nyrefunksjon og anbefalinger knyttet til dette.

Senere tok vi kontakt med farmasøyten fra hospiteringen og spurte om han kunne være vår faglige ekspert for vurdering av legemiddelgjennomgangene under forsøket. Dette ville han gjerne og han hjalp oss også med å utarbeide deler av spørreskjema for våre to kasus til forsøket. 

\section{HelsIT}
HelsIT er en årlig konferanse ment for å spre kunnskap og erfaring om bruk av IT i helsesektoren som fremmer det beste for samfunnet og innbyggere. Dette gjøres gjennom flere ulike arenaer og virkemidler, blant annet plenumsforedrag, seminarer, debattpaneler og mer. NTNU har ansvaret for konferansen, og styret består av relevante fagmiljøer ved universitetet samt andre samarbeidspartnere.  Deltagerne på HelsIT er fra ulike fagfelt og institusjoner. Målgruppen er 
\begin{itemize}
\item Utøvere og forvaltere av helsetjenester
\item Utviklere og leverandører av helserelaterte IT-systemer
\item Brukere av helsetjenester 
\item Akademia
\end{itemize}

Vi deltok på HelsIT 2015 for å primært få verdifull tilbakemelding fra fagfolk om vår plan for utførelse av masteroppgaven, samt høste erfaringer og komme i kontakt med næringslivet. Veileder Øystein Nytrø var leder av årets programkomité, så han hadde tanker om hvordan vi kunne involvere oss på konferansen.

%TODO: Sett inn vedlegg for presentasjonen på HelsIT
Nytt for HelsIT dette året var lanseringen av Innovathon - et verksted hvor mindre og større aktører innen helse-IT kunne få anledning til å presentere pilotløsninger og fremme nye ideer. På selve konferansen var dette en serie presentasjoner. Etter presentasjonene fikk alle bidragsytere tilbakemelding for sine ideer og løsninger. Formålet er å minske veien mellom teknologi, innovasjon og helsevesen. Veileder mente at dette var en ypperlig mulighet for å eksponere våre planer for masteroppgaven. På denne måten kunne vi få tilbakemelding fra interessenter, samt kanskje få kontakter som kunne bistå med relevant data og annet. Vi meldte oss på Innovathon og presenterte vår idé for masteroppgaven første dagen av konferansen\todo{vedlegg for presentasjonen som ref?}. Hvilket publikum det var under presentasjonene er uvisst, siden alle deltakere av konferansen hadde mulighet til å få det med seg Innovathon. Mest sannsynlig var dette personer som hadde direkte interesse av nyskapning og innovasjon innen helse-IT.  

Mye av tilbakemeldingen var preget av forvirring rundt det vi presenterte. Her var det fagfolk innen helse som ikke hadde den tekniske forståelsen vi har. Presentasjonen var i overkant teknisk med tema som semantisk web, så det hadde vi full forståelse for. Videre virket det som om publikum var usikker på hvilken tilbakemelding vi ønsket siden vi ikke hadde hverken en forretningsidé eller en ferdig løsning. Her måtte det presiseres at vi er en gruppe studenter som nylig startet på masteroppgaven; Ikke grundere med en forretningsidé. Vi ønsket tilbakemelding på hvordan vi tenkte å gjennomføre oppgaven. 

%TODO: Referer til tidligere kapittel om LMG i Midt-Norge
Leger i salen anbefalte å ta med en overlege som en del av prosessen, muligens som en biveileder. De mente at leger ville være mer nyttig enn farmasøyter for oss, siden leger tar beslutningene ved legemiddelgjennomgang. Her må vi ta i betraktning at legemiddelgjennomgang gjøres som allerede nevnt med ulike helsepersonell involvert i landet. Vi tror dermed ikke at legene her var kjent med situasjonen i Midt-Norge. Dette var også reflektert i en diskusjon vi hadde med sykepleiere fra Helse Sør-øst, som mente vi burde fokusere på sykehjem der mye av legemiddelgjennomgangen foretas. Siden vi har god kontakt med Sykehusapotekene i Midt-Norge med mulige biveiledere, vil kliniske farmasøyters arbeid i legemiddelgjennomganger være i fokus for oss.

%TODO: Referer til tidligere kapittel om LMH
Vi kom i kontakt med Aleksander Skøyeneie fra Legemiddelverket, som har mye innblikk i hvordan femstjerners FEST vil se ut. Her kom det fram at vi hadde mulighet til å få semistrukturerte data om legemidler fra legemiddelverket som ikke eksisterer i FEST, i hovedsak bivirkninger og indikasjoner. Legemiddelhåndboka innehar mye slik informasjon, men er ikke tilgjengelig i et strukturert format og derfor problematisk å anvende. Data fra legemiddelverket vil være til stor hjelp for oss, og vi kommer til å holde kontakten.

Deltagelsen på HelsIT 2015 var en nyttig erfaring. Her ble vi introdusert til domenet, knyttet ytterligere kontakt med næringslivet samt fikk god trening på å presentere vårt arbeid. Innovathon var kanskje ikke helt ment for oss. Som sagt hadde vi hverken en forretningsidé eller et produkt å vise fram. Med tanke på hvor langt vi er kommet med oppgaven var ikke tilbakemeldingene her av stor nytteverdi. 

%Tilbakemelding?
%-Leger anbefalte å ta med overlege som mulige veildere. Vi tror at de ikke er kjent med situasjonen i fylket! Vi kan fylle inn mye her. Grunnlag om hvorfor trøndelag gjør det på en slik måte og andre i landet ikke gjør det. 
%-Snakket med en fra Legemiddelverket(Aleksander) om mulighet for å få semistrukturerte data om legemidler(om bivirkninger osv som ikke er i FEST). Var veldig positiv. 
%-Vi tror ikke Innovathon var helt for oss. Vi er ikke grundere, og mye av tilbakemelding gikk nettopp på dette. 

%Refleksjon?    
%-Innovathon var ikke helt ment for oss, men vi fikk kontakt med næringslivet, nyttig erfaring. 

\section{Presentasjon for helsearbeidere}
Veileder Øystein Nytrø har kurs om klinisk beslutningsstøtte for helsearbeidere som tar etterutdanning. De hadde gått i gjennom tema som semantisk web, så her var det et publikum med kunnskapen vi trengte som også hadde en teknisk forståelse. Vi fikk mulighet til å presentere oppgaven vår, som i stor grad var lik innholdet i presentasjonen ved HelsIT. Her var det fagfolk som hadde kunnskap om legemiddelgjennomgang og klinisk beslutningsstøtte, så her håpet vi å få verdifull tilbakemelding.

Responsen på presentasjonen var positiv, og flere hadde spørsmål og tilbakemeldinger å komme med. Et spørsmål vi selv hadde problemer med å svare på, var hvorfor informasjonen på e-resept ikke er et godt nok datagrunnlag for å lage et beslutningsstøtte for legemiddelgjennomgang. Datagrunnlaget i e-resept er basert på FEST, som ikke har informasjon om bivirkninger og indikasjoner. Samtidig har ikke FEST en grunnleggende representasjon av legemidlene på et biokjemisk nivå. Her måtte veileder bistå, siden vi fortsatt ikke hadde nok kunnskap til å gi et godt svar. Et annet spørsmål fra publikum var om vi hadde tenkt på naturmedisiner. Dette er medisiner med ingredienser som ofte ikke er kartlagt i datakilder om legemidler. Siden naturmedisiner er i stadig økende forbruk er dette viktig å ta stilling til. Denne problemstillingen har vi ikke tenkt på tidligere, så dette var svært nyttig tilbakemelding for oss. Samtidig vet vi at svært mange ingredienser fra naturmedisiner er kartlagt i DrOn, ved å inkludere flere ontologier som har et vokabular for nettopp disse.
